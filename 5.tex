\chapter{Reason}

    The whole of our phisolophy amounts to ``follow reason''. ``Reason'' (Definition \ref{def:reason}), nevertheless, is a higher-level concept, and to grasp its meaning one must first grasp its hierarchical roots. These are what we have been discussing about in the chapters above.

        \begin{definition}[Reason]
        \label{def:reason}
            Method of cognition that proceeds in accordance with facts, which are established, directly or indirectly, by observation.
            
            The faculty that:
            \begin{itemize}
                \item identifies and integrates the material provided by man's senses;
                \item enables man to discover the nature of existents — by virtue of its power to condense sensory information in accordance with the requirements of an objective mode of cognition;
                \item organizes perceptual units in conceptual terms by following the principles of logic.
            \end{itemize}
        \end{definition}

        \begin{remark}
            The latter formulation highlights the three elements essential to the faculty: its data, percepts; its form, concepts; its method, logic.
        \end{remark}

    Reason is the existence-oriented faculty. Accepting reason is accepting reality.

    \section{Emotions as a Product of Ideas}
    
        Let us begin by defining the nature of emotions and their relationship to ideas. What is the connection between feeling and thinking?
        
        A feeling or emotion is a response to an object one perceives (or imagines). The object by itself, however, has no power to invoke a feeling in the observer. It can do so only if he supplies two intellectual elements, which are necessary conditions of any emotion.
        
        First, the person must know in some terms what the object is. He must have some understanding or identification of it.
        
        Second, the person must evaluate the object.
        
        Emotions are states of consciousness with bodily accompaniments and with spiritual—intellectual—causes.
        
        This last factor is the basis for distinguishing "emotion" from "sensation". A sensation is an experience transmitted by purely physical means; it is independent of a person's ideas. By contrast, love, desire, fear, anger, joy are not simply products of physical stimuli. They depend on the content of the mind.
        
        There are four steps in the generation of an emotion: perccption (or imagination), identification, evaluation, response.
        
        Normally, only the first and last of these are conscious. The two intellectual steps, identification and evaluation, occur as a rule without the need of conscious awareness and with lightninglike rapidity.
        
        once a man has formed a series of valuejudgments, he automatizes them.
        
        Value-judgments are formed ultimately on the basis of a philosophic view of man and life—of oneself, of others, of the universe; such a view, therefore, conditions all one's emotions.
        
        Most people hold their views of man and life only implicitly, not explicitly
        
        What makes emotions incomprehensible to many people is the fact that their ideas are not only largely subconscious, but also inconsistent. This leads to the appearance of a conflict between thought and feelings.

    \section{Reason as Man's Only Means of Knowledge}
    
        Reason is a faculty of awareness; its function is to perceive that which exists by organizing observational data. And reason is a volitional faculty; it has the power to direct its own actions and check its conclusions, the power to maintain a certain relationship to the facts of reality. Emotion, by contrast, is a faculty not of perception, but of reaction to one's perceptions. This kind of faculty has no power of observation and no volition; it has no means of independent access to reality, no means to guide its own course, and no capacity to monitor its own relationship to facts.
        
        Emotions are automatic consequences of a mind's past conclusions. Feeling follows obediently. It has no power to question its course or to check its roots against reality. Only man's volitional, existence-oriented faculty has such power.
        
        Now, through a study of man's means of consciousness, this earlier discussion has been confirmed and completed. Metaphysics and epistemology unite. They unite in declaring that "emotions are not tools of cognition.
        
        The conclusion is clear: there is no alternative or supplement to reason as a means of knowledge. If one attempts to give emotions such a role, then he has ceased to engage in the activity of cognition.
        
        If an individual experiences a clash between feeling and thought, he should not ignore his feelings. He should identify the ideas at their base (which may be a time-consuming process); then compare these ideas to his conscious conclusions, weighing the conflicts objectively; then amend his viewpoint accordingly, disavowing the ideas he judges to be false
        
        The above indicates the pattern of the proper relationship between reason and emotion in a man's life: reason first, emotion as a consequence.

    \section{The Arbitrary as Neither True or False}

    \section{Certainty as Contextual}