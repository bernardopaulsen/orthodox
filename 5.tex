\chapter{Reason}

    The whole of our phisolophy amounts to ``follow reason''. ``Reason'' (Definition \ref{def:reason}), nevertheless, is a higher-level concept, and to grasp its meaning one must first grasp its hierarchical roots. These are what we have been discussing about in the chapters above.

        \begin{definition}[Reason]
        \label{def:reason}
            Method of cognition that proceeds in accordance with facts, which are established, directly or indirectly, by observation.
            
            The faculty that:
            \begin{itemize}
                \item identifies and integrates the material provided by man's senses;
                \item enables man to discover the nature of existents — by virtue of its power to condense sensory information in accordance with the requirements of an objective mode of cognition;
                \item organizes perceptual units in conceptual terms by following the principles of logic.
            \end{itemize}
        \end{definition}

        \begin{remark}
            The latter formulation highlights the three elements essential to the faculty: its data, percepts; its form, concepts; its method, logic.
        \end{remark}

    Reason is the existence-oriented faculty. Accepting reason is accepting reality.

    \section{Emotions as a Product of Ideas}
    
        Let us begin by defining the nature of emotions and their relationship to ideas. What is the connection between feeling and thinking?
        
        A feeling or emotion is a response to an object one perceives (or imagines). The object by itself, however, has no power to invoke a feeling in the observer. It can do so only if he supplies two intellectual elements, which are necessary conditions of any emotion.
        
        First, the person must know in some terms what the object is. He must have some understanding or identification of it.
        
        Second, the person must evaluate the object.
        
        Emotions are states of consciousness with bodily accompaniments and with spiritual—intellectual—causes.
        
        This last factor is the basis for distinguishing "emotion" from "sensation".

    \section{Reason as Man's Only Means of Knowledge}

    \section{The Arbitrary as Neither True or False}

    \section{Certainty as Contextual}