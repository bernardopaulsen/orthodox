\chapter{Concept-Fromation}

    Sensory material is the first step of knowledge. Nevertheless, human knowledge is a conceptual phenomena. Therefore, man has to conceptualize the information provided by the senses.

    Conceptualiaztion is what enables us to go beyond the knowledge of only concretes and generalize, indentify natural laws, understand what we observe. Man won't see all trees, but can however obtain knowledge about all trees.

    A conceptual faculty determines a species' method of cognition, action and survival. To understand human knowledge one must understand concepts: what they are, how they are formed and how they are used in the quest for knowledge.

    \section{Differentiation and Integration as the Means to a Unit-Perspective}

        First, lets overview the nature of a conceptual consciousness. After man has indeitified particular entities, he can grasp relationships among these entities by grasping the similarities (Definition \ref{def:similarity}) and differences (Definition \ref{def:difference}) of their identities. These similarities are, in fact, observed in reality (Theorem \ref{the:similarities}).

            \begin{definition}[Similarity]
            \label{def:similarity}
                Partial identity.
            \end{definition}

            \begin{definition}[Difference]
            \label{def:difference}
                Partial non identity.
            \end{definition}

            \begin{theorem}
            \label{the:similarities}
                Similarities and differences are perceptually given.
            \end{theorem}

            \begin{proof}
                Just like the identity of an existence is observed in reality, is perceptually given, so are partial identities and partial non identities between existents.
            \end{proof}

        When we look at something, we do not se a thing, we see a thing of a specific kind, in relatioship to every other thing of the same kind (for example, you don't see this thing, you see this book, this object, this document). We grasp an entity as a member of a group of similar members. The implicit concept present in this view of the world is "unit", as in Denifition \ref{def:unit}. The ability to regard entities as units is man's distinctive method of cognition.

            \begin{definition}[Unit]
            \label{def:unit}
                An existent regarded as a separate member of a group of two or more similar members.
            \end{definition}

        By treating entities as members of groups of similar entities, man can apply to all entities of a group the knowledge he gains by studying only some of its members. It is essential to grasp that in the world apart form men there are no units, there are only existents. To view things as units is to adopt a human perspective on things.

        The concept "unit" is an act of consciousness, but it is not an arbitraty creatin of consciousness: it is a method of identification or classification according to the attributes which a consciousness observes in reality. Units do no exsit qua units, what exists are things, but units are things viewed by a consciousness in certain existing relationships.

        Without the implicit concept of "unit" man could not reach the conceptual method of knowledge. Without the same implicit conpect man could not enter the field of mathematics. Thus the same concept is the base and start of two fields: the conceptual and the mathematical, a fact that points to an essential connection between the two fields. It suggests that concept-formation is in some way a mathematical process.

        Two main processes are involved in the concious process man perform in order to regard entities as units: differentiation and integration (Definitions \ref{def:differentiation} and \ref{def:integration}).

            \begin{definition}[Differentiation]
            \label{def:differentiation}
                The pocess of grasping differences.
            \end{definition}

            \begin{definition}[Integration]
            \label{def:integration}
                The process of uniting elements into an inseparable whole.
            \end{definition}

        In order to move from the stage of sensation to that of perception, we have to discriminate sensory qualities and integrate them into entities. The same two processes occur in the movement from percepts to concepts. In this case, however, the processes are not performed for us automatically.

        We begin the formation of concepts by isolating a group of concretes. We do this on the basis of observed similarities that distibguish this concretes from the rest of our perceptual field. It is important to stress that the similarities are observed, they are perceptually given (Theorem \ref{the:similarities}).

        \begin{theorem}
        \label{the:similarities}
            Similarities are perceptually given.
        \end{theorem}

        \begin{proof}
            Following Definition \ref{def:similarity}, similarity is partial identity. Identity is self-evident, therefore perceptually given.
        \end{proof}

        The distinctively human element in the above is our ability to abstract (Definition \ref{def:abstraction}) such similarities from the differences in which they are embedded.

            \begin{definition}[Abstraction]
            \label{def:abstraction}
                The power of selective focus and treatment; it is the power to separate mentally and make cognitive use of an aspect of reality that cannot exist separately.
            \end{definition}

        Man makes something out of the similarities he observes: he makes such data the basis of a method of cognitive organization. The first step of the method is the mental isolation of a group of similars.

        But an isolated perceptual group is not yet a concept. To achieve a cognitive result, we must proceed to integrate. When we integrate entities into an iseparable whole such a whole is a new entitiy, a mental entity. This entity stands for an unlimited number of concretes, including the ones we have not yet observed.

        The tool that makes this kind of integration possible is language. A word is the only form in which man's mind is able to retain such a sum of concretes. A word (Definition \ref{def:word}) is a concrete, perceptually graspable symbol.

            \begin{definition}[Word]
            \label{def:word}
                A symbol that denotes a concept.
            \end{definition}

        Only concretes exist. If a concept is to exist, therefore, it must exist in some way as a concrete. It is not true that words are necessary primarily for the sake of communication. Words are essential to the process of conceptualization and thus to all thought. A word without a concept is noise. Words transform concepts into (mental) entities, definitions provide them with identity.

        Now let us consider a important problem: the relationship of concepts to existents. A percept is a direct awareness of an existing entity, but a concept involves a process of abstraction. A concept refers to what all the concretes in a given class possess in commom. The problem is: what is this attribute and how does one discover it?

        All along we have been using concepts to reach the truth. Now we must turn to the precondition of this use and face the fundamental problem of epistemology. We must ground concepts themselves in the nature of reality.

    \section{Concept-Formation as a Mathematical Process}

            \begin{definition}[Mathematics]
            \label{def:mathematics}
                Science of measurement.
            \end{definition}

            \begin{definition}[Measurement]
            \label{def:measurement}
                The identification of a relationship — a quantitative relationship established by means of a standard that serves as a unit.
            \end{definition}

        Measurement involes two concretes: the existent being measured and the existent that is the standard of measurement. In the process of measurement, we identify the relationship of any instance of a certain attribute to a specific instance of it selected as the unit. The former may range across the entire spectrum of magnitude; the latter, the (primary) unit, must be within the range of human perception.

        Similar concretes integrated by a concept differ from one another only quantitatively, only in the measurements of their characteristics. When we form a concept, therefore, our mental process consists in retaining the characteristics, but omitting their measurements. 
        
        The principle is: the relevant measurements must exist in some quantity, but may exist in any quantity. In this sense, in the form of an epistemological standing order, the concept may be said to retain all the characteristics of its referents and to omit all the measurements. A man's grasp of similarity is actually his mind's grasp of a mathematical fact: the fact that certain concretes are commensurable — that they (or their attributes) are reducible to the same unit(s) of measurement.

        So far, we have been considering measurement primarily in regard to the integration of concretes. Measurement also plays a special role in the first step of concept-formation: the differentiation of a group from other things. Such differentiation cannot be performed arbitrarily, only by a commensurable characteristic. 
        
        To differentiate existents, we define the concept of Concetual Commom Denominator (Definition \ref{def:ccd}). We also define concept (Definition \ref{def:concept}).

            \begin{definition}[Conceptual Common Denominator]
            \label{def:ccd}
                The characteristic(s) reducible to a unit of measurement, by means of which man differentiates two or more existents from other existents possessing it.
            \end{definition}

            \begin{definition}[Concept]
            \label{def:concept}
                A mental integration of two or more units possessing the same distinguishing characteristic(s), with their particular measurements omitted.
            \end{definition}

        A concept is not a product of arbitraty choice, it has a basis on and do refer to the facts of reality. A concept denotes facts - as processed by a human consciousness.

            \begin{theorem}
                A concept has a basis on and do refer to facts of reality.
            \end{theorem}

            \begin{proof}
                Perception makes us aware of the existents (facts of reality). Once we observe similarities (which are also facts of reality) between existents, we are able to differentiate them from all other existents, and finally integrate them into an indivisible whole.
            \end{proof}

        What the window of mathematics reveals is not the mechanics of deduction, but of induction. 

    \section{Definition as the Final Step in Concept-Formation}

        The final step in concept-formation is definition.

        The perceptual level of consciousness is automatically related to reality. A concept, however, is an integration that rests on a process of abstraction. Such a mental state is not automatically related to concretes, as is evident from the many obvious cases of ``floating abstractions'' (Definition \ref{def:floating}).

            \begin{definition}[Floating Abstraction]
            \label{def:floating}
                A concept detachets from existents.
            \end{definition}

        If a concept is to be a device of cognition, it must be tied to reality. It must denote units that one has methodically isolated from all others. This is the basic function of a definition: to distinguish a concept from all other concepts and thus to keep its units differentiated from all other existents.

        A definition cannot list all the characteristics of the units. Instead, a definition identifies a concept's units by specifying their essential characteristics (Definition \ref{def:essential_characteristic}) 

            \begin{definition}[Essential Characteristic(s)]
            \label{def:essential_characteristic}
                The fundamental characteristic(s) which makes the units the kind of existents they are and differentiates them from all other known existents.
            \end{definition}

        The distinguishing characteristic of an entity is the \textit{differentia}, the concetres from which we are distinguishing the entity from give rise to the \textit{genus}.

        Another such feature is the fact that definitions, like concepts, are contextual (Definition \ref{def:context}). The concepts' definitions may change as the context changes.

            \begin{definition}[Context]
            \label{def:context}
                The entire field of a mind's awareness or knowledge at any level of its cognitive development.
            \end{definition}

        When a definition is contextually revised, the new definition does not contradict the old one. The knowledge earlier gained remains knowledge. What changes is that, as one's field of knowledge expands, these facts no longer serve to differentiate the units.

        Definitions (like all truths) are  empirical statements. They derive from certain kinds of observations — those that serve a specific (differentiating) function within the conceptualizing process.

        The definition must state the feature that most significantly distinguishes the units; it must state the fundamental. 

            \begin{definition}[Fundamental]
            \label{def:fundamental}
                The characteristic responsible for all the rest of the units' distinctive characteristics, or at least for a greater number of these than any other characteristic is.
            \end{definition}

        The definitional principle is: wherever possible, an essential characteristic must be a fundamental.

        The truth of a proposition depends not only on its relation to the facts of the case, but also on the truth of the defi- nitions of its constituent concepts. If these concepts are detached from reality then so are the propositions that employ them. A proposition can have no greater validity than do the concepts that make it up. The precondition of the quest for truth, therefore, is the formulation of proper definitions.

        A concept is not interchangeable with its definition. A concept designates existents, including all their characteristics, whether definitional or not.

        It is crucially important to grasp the fact that a concept is an ``open-end'' classification which includes the yet-to-be-discovered characteristics of a given group of existents. One important implication of the above is that a concept, once formed, does not change. The knowledge men have of the units may grow and the definition may change accordingly, but the concept, the mental integration, remains the same. 






