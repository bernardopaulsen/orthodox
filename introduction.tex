\chapter{Introduction}

    The main goal of this introduction is to clarify some concepts that will be used throughout the book. We begin by defining ``self-evidence'' (Definition \ref{def:self_evidence}) and ``true'' (Definition \ref{def:true}).

        \begin{definition}[True]
        \label{def:true}
            In accordance with reality.
        \end{definition}

        \begin{definition}[Self-Evidence]
        \label{def:self_evidence}
            A proposition that is known to be true by sense perception, without the need of proof.
        \end{definition}

        \begin{example}
            You exist.
        \end{example}

    Now, we can define ``axiom'', ``corollary'', ``theorem'' and ``proof'' (Definitions \ref{def:axiom}, \ref{def:corollary}, \ref{def:theorem} and \ref{def:proof}).

        \begin{definition}[Axiom]
        \label{def:axiom}
            A self evidence.
        \end{definition}

        \begin{definition}[Corollary]
        \label{def:corollary}
            A self evidence that follows from another self-evidence.
        \end{definition}

        \begin{definition}[Theorem]
        \label{def:theorem}
            A proposition that is not self evident but it proved by a chain of reasoning; a truth established by means of accepted truths.
        \end{definition}

        \begin{definition}[Proof]
        \label{def:proof}
            Argument establishing the truth of a statement.
        \end{definition}

    This book is about philosophy. Below are the definitions of the five branches of philosohy the will be discussed in this book: metaphysics, epistemology, ethics, politics and aesthetics (Definitions \ref{def:metaphysics}, \ref{def:epistemology}, \ref{def:ethics}, \ref{def:politics}, \ref{def:aesthetics}).

        \begin{definition}[Metaphysics]
        \label{def:metaphysics}
            The branch of philosophy that deals with the first principles of things.
        \end{definition}

        \begin{definition}[Epistemoly]
        \label{def:epistemology}
            The theory of knowledge, especially with regard to its methods, validity, and scope.
        \end{definition}

        \begin{definition}[Ethics]
        \label{def:ethics}
            The branch of knowledge that deals with moral principles.
        \end{definition}

        \begin{definition}[Politics]
        \label{def:politics}
            The academic study of government and the state.
        \end{definition}

        \begin{definition}[Aesthetics]
        \label{def:aesthetics}
            The branch of philosophy that deals with the principles of beauty and artistic taste.
        \end{definition}

    New concepts will be defined throughout.