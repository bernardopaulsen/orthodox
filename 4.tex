\chapter{Objectivity}

    We will now begin to identify the rules men must follow in their thinking if knowledge is the goal. These rules can be condensed into one principle: thinking, to be valid, must adhere to reality. But how does one guarantee adherence to reality? The answer lies in the concept of "objectivity".

    \section{Concepts as Objective}

        "Objectivity" arises because concepts are formed by a specific process and, as a result, bear a specific kind of relationship to reality. Concepts do not pertrain to consciousness alone or to existence alone, they are products of a specific kind of relationship between the two. Abstractions are products of man's faculty of cognition, which, concerned with grasping reality, must adhere to it.

        Conceptualization is not an automatic reaction to stimuli (as perception is). Concept formation is volitional, requiring effort. Man must learn to do it correctly. In such processing, the basic method he uses, measurement-omission, is dictated by the nature of his cognitive faculty. The result is a human perspective on things, not a revelation of a special sort of entity or attribute intrinsic in the world apart from man.

        On the other hand, consciousness is the faculty of grasping that which is, and there is a metaphysical basis for concepts. The charateristics of entities is a fact, not a creation of man. The method of concept formation conforms each step to facts, otherwise it would be irrelevant to a cognitive need.

    \section{Objectivity as Volitional Adherence to Reality by a Method of Logic}

        The objective approach to concepts leads to the view that, beyond the perceptual level, knowledge is the grasp of an object through an active, reality-based process chosen by the subject. The steps of this process must contitute a method of cognition, that guarantees that men remains in contact with reality. We define objectivity in Definition \ref{def:objectivity}. Reality, existents, cannot be objetive, they simply are. It is conceptual processes which are objective.

            \begin{definition}[Obectivity]
            \label{def:objectivity}
                Volitional adherence to reality by following certain rules of method, a method based on facts and appropriate to man's form of cognition.
            \end{definition}

        The method of cognition that objectivity requires is logic (Definition \ref{def:logic}), which is a volitional consciousness' method of conforming to reality, it is the method of reason. Logic is the art of noncontradictory identification.

    \section{Knowledge as Contextual}

    \section{Knowledge as Hierarchical}

    \section{Intrinsicism and Subjectivism as the Two Forms of Rejecting Objectivity}




