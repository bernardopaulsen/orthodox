\part{Metaphysics}

\chapter{Reality}

    \section{Existence, Identity and Consciousness as the Basic Axioms}

        The first thing we perceive when we are awake is very simple: something exists, there is something. It does not matter what you focus on, nor if your eyes are open or closed; if you perceive something, you perceive that something exists. We start from exactly this self-evidence. Thus the first (and only) axiom of philosophy is Axiom \ref{ax:existence}.
    
            \begin{axiom}[Existence]
            \label{ax:existence}
                Existence exists.
            \end{axiom}
            
        As something self-evident, existence requires no proof. Also, the concept ``existence'' cannot be defined by means of other concepts. As the widest of all concepts, referring to everything which there is, it can be defined only ostensively (by pointing at existents, for example).

        To be is to be something (specific). To exist is to possess an identity (as in Definition \ref{def:identity}). This leads us to Corollary \ref{cor:identity}.
            
            \begin{definition}[Identity]
            \label{def:identity}
                \begin{enumerate}
                    \item The fact of being who or what a person or thing is.
                    \item Sum of a person or thing's attributes or characteristics.
                \end{enumerate}
            \end{definition}
        
            \begin{axiom}[Law of Identity]
            \label{cor:identity}
                To exist is to posses identity, to be is to be something. Each thing is identical with itself.
            \end{axiom}

            \begin{remark}
                Identity is a corollary of Existence.
            \end{remark}
    
        The fact that we perceive the existence of something, whatever it may be, makes us possess consciousness, following Definition \ref{def:consciousness}. This leads us to Corollary \ref{cor:consciousness}.
        
            \begin{definition}[Consciousness]
            \label{def:consciousness}
                The faculty of being aware of that which exists.
            \end{definition}
            
            \begin{corollary}[Consciousness]
            \label{cor:consciousness}
                Consicousness exists.
            \end{corollary}

            \begin{remark}
                Consciousness is a corollary of Law of Identity.
            \end{remark}
            
        Consciousness is impossible without existence: there would be nothing to be conscious of. There is never consciousness of something which does not exist. If something nonexistent is perceived than what the perceiver possess is not consciousness. Just as the first and only axiom (Axiom \ref{ax:existence}) the first two corollaries are also known to be true merely by sense perception, without the possibility for or need of any proof. As a matter of fact, every proof that exists takes the three axioms above as given.
        
        Two corollaries are implicit in the Law of Identity: the Law of Non-Contradiction (Corollary \ref{cor:non}) and the Law of the Excluded Middle (Corollary \ref{cor:excluded}). The Law of Identity together with its two corollaries define what we know as logic (Definition \ref{def:logic}).
            
            \begin{corollary}[Law of Non-contradiction]
            \label{cor:non}
                Something never is and is not at the same time.
            \end{corollary}

            \begin{remark}
                Law of Non-Contradiction is a corollary of Law of Identity.
            \end{remark}
            
            \begin{corollary}[Law of Excluded Middle]
            \label{cor:excluded}
                Something either is or is not, there is no third alternative.
            \end{corollary}

            \begin{remark}
                Law of Excluded-Middle is a corollary of Law of Non-Contradiction.
            \end{remark}

            \begin{definition}[Logic]
            \label{def:logic}
                Set of three laws:
                \begin{itemize}
                    \item Law of Identity,
                    \item Law of Non-Contradiction,
                    \item Law of the Excluded Middle.
                \end{itemize}
            \end{definition}
            
        Logic is an indispensable characteristic of existence. Absolutely everything that exists is bound to the laws of logic.
            
    \section{Causality as a Corollary of Identity}
        
        The content of the world men perceive is constituted by entities, which are everything there is to observe. The concept of ``entity'', just like the concept of ``existence'', cannot be defined with other concepts, just ostensively. Categories of being (such as qualities or quantities) do not have metaphysical primacy, as all represent merely aspects of entities.

        Actions, as in Definition \ref{def:action}, also do not exist apart from entities. There are no floating actions: there are only actions performed by entities. After we also define nature (Definition \ref{def:nature}) we arrive to the Law of Causality (Corollary \ref{cor:causality}).
        
            \begin{definition}[Action]
            \label{def:action}
                What entities do.
            \end{definition}
            
            \begin{definition}[Nature]
            \label{def:nature}
                The attributes of an entity, what an entity is.
            \end{definition}
        
            \begin{corollary}[Law of Causality]
            \label{cor:causality}
                An entity acts in accordance with its nature.
            \end{corollary}

            \begin{remark}
                Law of causality is a corollary of identity. If an action is what an entity does, and its nature is its attributes, than the action, being an attribute of the entity, is in accordance with its nature. For the law of causality to not hold, an entity would need to act apart from its nature or act against it, both of which are impossible given the Law of Identity (Corollary \ref{cor:identity}). Apart from its nature, its attributes, a thing is nothing.
            \end{remark}
            
        The action of an entity is both caused (Definition \ref{def:cause}) and necessitated (Definition \ref{def:necessity}) by its nature. (Theorem \ref{the:cause_necessity}).

            \begin{definition}[Cause]
            \label{def:cause}
                \begin{enumerate}
                    \item A person or thing that gives rise to an action, phenomenon, or condition.
                    \item reasonable grounds for doing, thinking, or feeling something.
                \end{enumerate}
            \end{definition}
        
            \begin{definition}[Necessity]
            \label{def:necessity}
                Principle according to which something must be so.
            \end{definition}

            \begin{theorem}
            \label{the:cause_necessity}
                The action of an entity is both caused and necessitated by its nature.
            \end{theorem}

            \begin{proof}
                The nature of the entity (what it is) is what causes the entity's actions. The actions must be so, as an entity acts in accordance with its nature.
            \end{proof}
            
        Cause and effect are an universal law of reality. It is a law inherent in being qua being: to be is to be something (Corollary \ref{cor:identity}), and to be something is to act accordingly (Corollary \ref{cor:causality}). Every action has a cause, and the same cause leads to the same effect. Nevertheless, not every entity has a cause. The universe, being everything that exists, cannot have a cause, as this would require it to be caused by nonexistence, where there is nothing to act as a cause.
        
    \section{Existence as Possessing Primacy over Consciousness}
        
        Existence precedes consciousness. Consciousness, following Definition \ref{def:consciousness}, is the faculty of being aware. An entity's nature is not affected by the nature of any other entity, even if the latter's nature includes the faculty of being aware of the first's existence. This fact is summarized in Corollary \ref{cor:primacy}.
        
            \begin{corollary}[Primacy of Existence]
            \label{cor:primacy}
                Things are what they are independent of consciousness.
            \end{corollary}
        
        Proofs depend on the primacy of existence, as they presuppose the principle that facts are not malleable. Since knowledge is knowledge of reality, every metaphysical principle has epistemological implications. As the nature of the world is not maleable, nothing is relevant to cognition except data drawn from the world. This leads us to the only valid method of cognition: reason.
            
    \section{The Metaphysically Given as Absolute}
    
        The discussion above culminates in the principle that no alternative to a fact of reality is possible. The metaphysically given is absolute (Corollary \ref{cor:given}).
    
            \begin{definition}[Metaphysically Given]
                Any fact inherent in existence apart from human action.
            \end{definition}
            
            \begin{definition}[Absolute]
                Necessitated by the nature of existence.
            \end{definition}

            \begin{definition}[Necessity]
                Principle according to which something must be so.
            \end{definition}
            
            \begin{corollary}
            \label{cor:given}
                The metaphysically given is absolute.
            \end{corollary}
            
        The antonym of ``necessary'' is ``chosen'', which is the case of man-made facts. Creativity (Definition \ref{def:creativity}) alters reality, but not the metaphysically given. In order for man to succeed, he needs to accept the absolute.

            \begin{definition}[Creativity]
            \label{def:creativity}
                The use of the imagination.
            \end{definition}

            \begin{definition}[Imagination]
                The faculty or action of forming new ideas, or images or concepts of external objects not present to the senses.
            \end{definition}
            
        The distinction between the given and the man-made is crucial. The given is reality, and must be accepted without evaluation. Man-made facts, on the contrary, are products of choice, and must be evaluated.